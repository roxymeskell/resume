\documentclass[10pt]{article} % Don't like 10pt? Try 11pt or 12pt
\usepackage{calc} % This is a helpful package that puts math inside length specifications
\usepackage{etoolbox}
\usepackage{multicol} % Allows multicolumn formatting
\reversemarginpar % Puts the section titles on left side of page
\usepackage[paper=letterpaper,
			marginparwidth=1.4in,     % Length of section titles
			marginparsep=.05in,     % Space between titles and text
			margin=.4in,            % 1 inch margins
			includemp]{geometry}
\usepackage[shortlabels]{enumitem}
\usepackage{fancyhdr}
\usepackage{color,hyperref}


\usepackage{resutil}
\usepackage{projects}
\usepackage{workexp}
\usepackage{extracurr}

%\usepackage{pdftexcmds} % Package for switch statements

\definecolor{darkblue}{rgb}{0.0,0.0,0.3}
\hypersetup{breaklinks,colorlinks,
	linkcolor=black,urlcolor=black,
	anchorcolor=black,citecolor=black,
}
\pagestyle{empty}
\setlength{\parindent}{0in} % Get rid of indenting throughout entire document


% The OCR software also has a hard time with italics. These commands get
% rid of the two common ways to italicize text in LaTeX. Get rid of them
% to turn italics back on.
\renewcommand\emph[1]{#1}
\renewcommand\textit[1]{\underline{#1}}


%%%%%%%%%%%%%%%%%%%%%%%% End Document Setup %%%%%%%%%%%%%%%%%%%%%%%%%%%%

%%%%%%%%%%%%%%%%%%%%%%%%%%% Helper Commands %%%%%%%%%%%%%%%%%%%%%%%%%%%%

%%% HEADING AT TOP

% The title (name) with a horizontal rule under it
% (optional argument typesets an object right-justified across from name
%  as well)
%
% Usage: \makeheading{name}
%        OR
%        \makeheading[right_object]{name}
%
% Place at top of document. It should be the first thing.
% If ``right_object'' is provided in the square-braced optional
% argument, it will be right justified on the same line as ``name'' at
% the top of the CV. For example:
%
%       \makeheading[\emph{Curriculum vitae}]{Your Name}
%
% will put an emphasized ``Curriculum vitae'' at the top of the document
% as a title. Likewise, a picture could be included:
%
%   \makeheading[\includegraphics[height=1.5in]{my_picutre}]{Your Name}
%
% the picture will be flush right across from the name.
\newcommand{\makeheading}[2][]%
{\hspace*{-\marginparsep minus \marginparwidth}%
	\begin{minipage}[t]{\textwidth+\marginparwidth+\marginparsep}%
		{\LARGE \bfseries #2 \hfill #1}\\[-0.15\baselineskip]%
		\rule{\columnwidth}{1pt}%
	\end{minipage}}

%%% SECTION HEADINGS

% The section headings. Flush left in small caps down pseudo-margin.
%
% Usage: \section{section name}
\renewcommand{\section}[1]{\pagebreak[3]%
	%\vspace{1\baselineskip}%
	\par %\vspace*{0.5\baselineskip}\startsection
	%\startsection\vspace*{0.5\baselineskip}
	\phantomsection\addcontentsline{toc}{section}{#1}%
	\noindent\llap{\scshape\smash{\parbox[t]{\marginparwidth}{\hyphenpenalty=10000\raggedright #1}}}%
	\vspace{-\baselineskip}\par}

%%% COLUMNS
\newenvironment{columns}[1]
{
	\newcounter{columnCount}
	\newenvironment{column}[1]
	{
		\textbf{##1}
	}
	{
		\ifnumcomp{\value{columnCount}}{>}{#1-1}{}{\columnbreak}
		\stepcounter{columnCount}
	}
	\newenvironment{listcolumn}[1]
	{
		\textbf{##1}
		\begin{innerlist}
	}
	{
		\end{innerlist}
		\ifnumcomp{\value{columnCount}}{>}{#1-1}{}{\columnbreak}
		\stepcounter{columnCount}
	}
	\begin{multicols}{#1}
}
{
	\end{multicols}
}

%%%% LISTS
%
%% This macro alters a list by removing some of the space that follows the list
%% (is used by lists below)
%\newcommand*\fixendlist[1]{%
%	\expandafter\let\csname preFixEndListend#1\expandafter\endcsname\csname end#1\endcsname
%	\expandafter\def\csname end#1\endcsname{\csname preFixEndListend#1\endcsname\vspace{-0.6\baselineskip}}}
%
%% These macros help ensure that items in outer-type lists do not get
%% separated from the next line by a page break
%% (they are used by lists below)
%\let\originalItem\item
%\newcommand*\fixouterlist[1]{%
%	\expandafter\let\csname preFixOuterList#1\expandafter\endcsname\csname #1\endcsname
%	\expandafter\def\csname #1\endcsname{\csname preFixOuterList#1\endcsname\let\oldItem\item\def\item{\pagebreak[2]\oldItem}}
%	\expandafter\let\csname preFixOuterListend#1\expandafter\endcsname\csname end#1\endcsname
%	\expandafter\def\csname end#1\endcsname{\let\item\oldItem\csname preFixOuterListend#1\endcsname}}
%\newcommand*\fixinnerlist[1]{%
%	\expandafter\let\csname preFixInnerList#1\expandafter\endcsname\csname #1\endcsname
%	\expandafter\def\csname #1\endcsname{\let\oldItem\item\let\item\originalItem\csname preFixInnerList#1\endcsname}
%	\expandafter\let\csname preFixInnerListend#1\expandafter\endcsname\csname end#1\endcsname
%	\expandafter\def\csname end#1\endcsname{\csname preFixInnerListend#1\endcsname\let\item\oldItem}}

% An itemize-style list with lots of space between items
%
% Usage:
%   \begin{outerlist}
%       \item ...    % (or \item[] for no bullet)
%   \end{outerlist}
\newlist{outerlist}{itemize}{3}
\setlist[outerlist]{label=\enskip\textbullet,leftmargin=*}
\fixendlist{outerlist}
\fixouterlist{outerlist}

% An environment IDENTICAL to outerlist that has better pre-list spacing
% when used as the first thing in a \section
%
% Usage:
%   \begin{lonelist}
%       \item ...    % (or \item[] for no bullet)
%   \end{lonelist}
\newlist{lonelist}{itemize}{3}
\setlist[lonelist]{label=\enskip\textbullet,leftmargin=*,partopsep=0pt,topsep=0pt}
\fixendlist{lonelist}
\fixouterlist{lonelist}

% An itemize-style list with little space between items
%
% Usage:
%   \begin{innerlist}
%       \item ...    % (or \item[] for no bullet)
%   \end{innerlist}
\newlist{innerlist}{itemize}{3}
\setlist[innerlist]{label=\enskip\textbullet,leftmargin=*,parsep=0pt,itemsep=0pt,topsep=0pt,partopsep=0pt}
\fixinnerlist{innerlist}

% An environment IDENTICAL to innerlist that has better pre-list spacing
% when used as the first thing in a \section
%
% Usage:
%   \begin{loneinnerlist}
%       \item ...    % (or \item[] for no bullet)
%   \end{loneinnerlist}
\newlist{loneinnerlist}{itemize}{3}
\setlist[loneinnerlist]{label=\enskip\textbullet,leftmargin=*,parsep=0pt,itemsep=0pt,topsep=0pt,partopsep=0pt}
\fixendlist{loneinnerlist}
\fixinnerlist{loneinnerlist}

%% List for projects
%%
%%
%\newlist{projectlist}{itemize}{3}
%\setlist[projectlist]{label=\enskip\textendash,leftmargin=0.4in,parsep=0pt,itemsep=0pt,topsep=0pt,partopsep=0pt}
%\fixinnerlist{projectlist}

%%% EXTRA SPACE

% To add some paragraph space between lines.
% This also tells LaTeX to preferably break a page on one of these gaps
% if there is a needed pagebreak nearby.
\newcommand{\blankline}{\quad\pagebreak[3]}
\newcommand{\halfblankline}{\quad\vspace{-0.5\baselineskip}\pagebreak[3]}

%%% FORMATTING MACROS

% Uses hyperref to link DOI
\newcommand\doilink[1]{\href{http://dx.doi.org/#1}{#1}}
\newcommand\doi[1]{doi:\doilink{#1}}

% For \url{SOME_URL}, links SOME_URL to the url SOME_URL
\providecommand*\url[1]{\href{#1}{#1}}
% Same as above, but pretty-prints SOME_URL in teletype fixed-width font
\renewcommand*\url[1]{\href{#1}{\texttt{#1}}}

% For \email{ADDRESS}, links ADDRESS to the url mailto:ADDRESS
\providecommand*\email[1]{\href{mailto:#1}{#1}}
% Same as above, but pretty-prints ADDRESS in teletype fixed-width font
%\renewcommand*\email[1]{\href{mailto:#1}{\texttt{#1}}}

%\providecommand\BibTeX{{\rm B\kern-.05em{\sc i\kern-.025em b}\kern-.08em
%    T\kern-.1667em\lower.7ex\hbox{E}\kern-.125emX}}
%\providecommand\BibTeX{{\rm B\kern-.05em{\sc i\kern-.025em b}\kern-.08em
%    \TeX}}
\providecommand\BibTeX{{B\kern-.05em{\sc i\kern-.025em b}\kern-.08em
		\TeX}}
\providecommand\Matlab{\textsc{Matlab}}

% Custom hyphenation rules for words that LaTeX has trouble with
\hyphenation{bio-mim-ic-ry bio-in-spi-ra-tion re-us-a-ble pro-vid-er}



%%%%%%%%%%%%%%%%%%%%%%%% End Helper Commands %%%%%%%%%%%%%%%%%%%%%%%%%%%

\begin{document}
	%%\thispagestyle{plain}
	\makeheading{Roxanna Meskell}
	%
	\vspace*{0.5\baselineskip}\section{Contact Information}
	
	% NOTE: Mind where the & separators and \\ breaks are in the following
	%       table. Table is one row made up of three parboxes. The left
	%       parbox has address info, the middle parbox has a vertical bar,
	%       and the right parbox has phone and electronic contact
	%       information.
	%
	% MACROS: \rcollength is the width of the right column of the table
	%             (adjust it to your liking; default is 1.85in).
	%         \spacewidth is width of area between left and right boxes.
	%         \spacechar is character used to produce perforated vertical
	%             boundary between boxes.
	%
	\newlength{\rcollength}\setlength{\rcollength}{2in}%
	\newlength{\spacewidth}\setlength{\spacewidth}{20pt}
	\newcommand\spacechar{$|$}
	%
	\begin{tabular}[t]{@{}p{\textwidth-\rcollength-\spacewidth}@{}p{\spacewidth}@{}p{\rcollength}}%
		
		% Address box
		\parbox{\textwidth-\rcollength-\spacewidth}{%
			2380A Scottsville Rd.\\
			Scottsville, NY 14546\\}
		
		% Cheesy perforated vertical bar between boxes
		% Shorten by removing \spacechar's
		& \parbox{\spacewidth}{\centering} &
		
		% Non-snail-mail contact information
		\parbox{\rcollength}{%
			%\emph{Phone:} 
			770-356-1249\\
			%\emph{E-mail:}
			\email{rlm1388@rit.edu}\\
			github.com/roxymeskell\\
		}
		
	\end{tabular}
	
	%%
	%% In modern CV's, it seems like ``Objective'' is frowned upon. Instead,
	%% incorporate it into a well-constructed cover letter. The ``More
	%% information'' can go at the end of the CV, but it should not distract
	%% from the section giving references available to contact.
	%%
	%
	\section{Objective}
	%Interested in a single or double block co-op for 2016 working as a software developer.
	Looking for a single or double block co-op for Spring, Summer, and/or Fall 2017 in a computer science field in order to expand my skill set and gain real world experience.
	
	\vspace*{0.5\baselineskip}\section{Education}
	
	\textbf{Rochester Institute of Technology}
	- Rochester, NY
	\hfill September 2013 - Present
	%\begin{outerlist}
		%\item []
		\begin{innerlist}
			\item Major: B.S. Computer Science
			\item Expected Graduation Date: June 2018
		\end{innerlist}
	%\end{outerlist}
	

	
	\vspace*{0.5\baselineskip}\section{Skills}
	\setlength{\columnsep}{4pc}
	\setlength{\multicolsep}{0.2pc}
	\begin{multicols}{3}
		\setlength{\multicolsep}{0.5pc}
		\setlength{\columnsep}{1pc}
		\textbf{Languages}:
		\begin{multicols}{2}
			\begin{loneinnerlist}
				\item C
				\item Java
				\item Javascript
				\item C\#
				%\item MIPS
				\item Haskell
				\item SQL
				\item Python
				\item HTML
				\item CSS
				%\item \LaTeX
			\end{loneinnerlist}
		\end{multicols}
		\columnbreak
		\textbf{Tools}:
		\begin{multicols}{2}
			\begin{loneinnerlist}
				\item gdb
				\item git
				\item vim
				\item grunt
				%\item postgreSQL
				\item Django
				%\columnbreak
				%\item Eclipse
				%\item Netbeans
				%\item Visual Studio
				\item nodeJS
				\item reactJS
				%\item Microsoft Office
				\item MatLab
				%\item Android\\Studio
			\end{loneinnerlist}
		\end{multicols}
		\columnbreak
		\textbf{Operating Systems}:
		\begin{multicols}{2}
			\begin{loneinnerlist}
				\item Windows
				\item Linux
				\item Mac OS X
				\item[]
				\item[]
				\columnbreak
				\item[]
				\item[]
				\item[]
				\item[]
				\item[]
			\end{loneinnerlist}
		\end{multicols}
	\end{multicols}
	
	\vspace{-0.5\baselineskip}
	
	\section{Projects}
	
	\fproj{Song Database}[Data Management][Fall 2014][‘Language: Java&PostgreSQL database&Group project&Stores information for songs and playlists&Wrote majority of the SQL code for the database’]
	\fproj{Threaded Multidimensional Maze Generator}[Personal][2014 - Ongoing][‘Language: C&Uses binary division to generate maze&Plans to be able generate a maze spanning at least 16 dimensions&Plans to be able to output maze to a file’]
	\fproj{Maze Generation Simulator}[Personal][2013 - Ongoing][‘Language: Java&Shows the progression of mazes as they are generated using different algorithms&Intended for use during a seminar’]
	\expandprojoff
	\fproj{Multidimensional Maze Generator}[Personal][2013 - 2014][‘Language: C\#&Uses Eller's Algorithm to randomly generate a maze spanning up to 16 dimensions&Maze is viewable and able to be transversed in the console’]
	\fproj{Ingredient Organizer App}[Personal][2015 - Ongoing][‘Language: Java&PostgresSQL database&For Android&Stores recipes and tracks ingredients user has on hand’]
	%\fproj{Maze Generation Simulator}[Personal][2013 - Ongoing][‘Language: C\#&Shows the progression of mazes as they are generated using different algorithms&Uses windows forms’]
	\fproj{Intelligent Clue Player}[Intelligent Systems][Spring 2015][‘Language: Java&Coded player for Clue to make the most optimal move and win nearly every game’]
	\fproj{Healthnet}[Software Engineering][Fall 2015][‘Language: Python, HTML, CSS&Uses Django&Group project&A website to track patient and medical staff information for hospitals’]
	\fproj{Assembly Sudoku Solver}[Concepts of Computer Systems][Fall 2015][‘Language: MIPS Assembly&A program to solve and display sudoku puzzles written completely in assembly.’]
	%
	%
	\section{Coursework}
	\vspace{-0.25\baselineskip}
	\begin{columns}{2}
		\begin{innerlist}
			\item Intro to Computer Vision
			\item Intelligent Systems
			\item Concepts of Parallel \& Distributed Systems
			\item Concepts of Computer Systems
			\item Analysis of Algorithms
			\item Mechanics of Programming
			\item Data Management
			\item Programming Language Concepts
			\item Computer Science Theory
			\item Software Engineering
			%\item Financial Accounting
			%\item Professional Communications
			%\item Multivar & Vector Calculus
			%\item Graph Theory
			%\item Principles of Marketing
			%\item Organizational Behavior
		\end{innerlist}
	\end{columns}
	
	\hyphenation{customers}
	\vspace{0.5\baselineskip}\section{Work Experience}
	\fwork{Intuit}[Mountain View, CA][January 2016 - July 2016][Software Engineering Co-op][‘Worked as part of the QuickBooks Online Payments team. Work primarily consisted of front-end feature development in javascript, as well as testing of current features.&Also worked on side project to create a browser based diagnostic tool using reactJS. {\it fixit.intuit.com}’]
	% Competed in a group with other co-ops to design a product from scratch. My team won the competition. Codechella 
	\vspace{-\baselineskip}
	\fwork{Siemens Industry}[Norcross, GA][June 2012 - July 2012][Business Intern]
	\section{Extracurricular}
	\expandexcoff%
	\fextracurr{Computer Science House (CSH)}[Fall 2013 - Present][][]
	\fextracurr{Society of Software Engineers (SSE)}[Fall 2015 - Present][][]
	\fextracurr{FIRST Robotics (Highschool)}[Fall 2011 - Spring 2013][][]

	\halfblankline
	
\end{document}

% UTILITIES NEEDED
% - Space between sections (\vspace{0.5\baselineskip})
% - Space between section chunks (\vspace{0.5\baselineskip})
% - Unspace between small section chunks (\vspace{-0.5\baselineskip})